\documentclass{article}

\setlength\parindent{0pt}

\title{Computational Physics - Linux and Command Line Skill Assessment}

\author {Samuel Barton}

\begin{document}
	
	\maketitle
	
	This is not a graded exercise, it is merely a test of your current ability to use Linux and the command line. If you cannot do any of the work with the command line, don't feel bad. By the end of the course you'll understand how to do all of this stuff, and why you'd care to.
\\
\\
	New you may well ask "What does this have to do with physics?", which is a very valid question. The answer is simple, knowledge of Linux and the command line will help you to be a more efficient scientist with a broad set of tools at your disposal. This class will teach you about programming in Python in order to solve physics problems, and basic command line skills to aid you in your programming endeavors.
	
	\newpage
	
	\begin{enumerate}
		\item Create a new directory in your home folder called \textit{physics}. (In the GUI).
		\item Change directories to \textit{physics}. 
		\item Open the terminal. 
		\item Make sure you are in your \textit{home} folder.
		\item Change directories to \textit{physics}
		\item Create a new directory from the terminal called \textit{proj1}.
		\item While still in the terminal, open a file in a text editor in the \textit{proj1} directory called \textsf{testing.txt}.
		\item Type the phrase ``Hello World!'' into the text editor and save the file.
		\item List the contents of the directory from the terminal.
		\item Search the physics directory for files containing the word ``Hello''. (you may want to google ``grep'')
		\item Install the ``vim'' text editor from the terminal.
		\item Copy \textsf{testing.txt} into your home folder.
		\item Move \textsf{testing.txt} from \textit{proj1} to its parent directory (\textit{physics}).
		\item Remove \textsf{testing.txt} from the home folder.
		\item Calculate 2 + 2 using the terminal.
		\item Show the detailed properties of the files in the physics directory.
		\item Display your computer's current performance statistics.
		\item Clear the screen of your terminal.
		\item Print out the current date and time on the terminal.
		\item Show all currently running processes in the terminal.
	\end{enumerate}
\end{document}